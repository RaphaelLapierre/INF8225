%%%%%%%%%%%%%%%%%%%%%%%%%%%%%%%%%%%%%%%%%
% Thin Sectioned Essay
% LaTeX Template
% Version 1.0 (3/8/13)
%
% This template has been downloaded from:
% http://www.LaTeXTemplates.com
%
% Original Author:
% Nicolas Diaz (nsdiaz@uc.cl) with extensive modifications by:
% Vel (vel@latextemplates.com)
%
% License:
% CC BY-NC-SA 3.0 (http://creativecommons.org/licenses/by-nc-sa/3.0/)
%
%%%%%%%%%%%%%%%%%%%%%%%%%%%%%%%%%%%%%%%%%

%----------------------------------------------------------------------------------------
%	PACKAGES AND OTHER DOCUMENT CONFIGURATIONS
%----------------------------------------------------------------------------------------

\documentclass[a4paper, 12pt]{article} % Font size (can be 10pt, 11pt or 12pt) and paper size (remove a4paper for US letter paper)

\usepackage[protrusion=true,expansion=true]{microtype} % Better typography
\usepackage{graphicx} % Required for including pictures
\usepackage[utf8]{inputenc}
\usepackage[margin=1.0in]{geometry}
\usepackage{url}
\usepackage{fancyhdr}
\usepackage{amsmath}
\usepackage{setspace}
\usepackage{enumitem}
\setlength\parindent{0pt} % Removes all indentation from paragraphs

\usepackage[T1]{fontenc} % Required for accented characters
\usepackage{times} % Use the Palatino font

\usepackage{listings}
\usepackage{color}
\lstset{mathescape}

\definecolor{dkgreen}{rgb}{0,0.6,0}
\definecolor{gray}{rgb}{0.5,0.5,0.5}
\definecolor{mauve}{rgb}{0.58,0,0.82}

\lstset{frame=tb,
   language=c++,
   aboveskip=3mm,
   belowskip=3mm,
   showstringspaces=false,
   columns=flexible,
   basicstyle={\small\ttfamily},
   numbers=none,
   numberstyle=\tiny\color{gray},
   keywordstyle=\color{blue},
   commentstyle=\color{dkgreen},
   stringstyle=\color{mauve},
   breaklines=true,
   breakatwhitespace=true
   tabsize=3
}
\linespread{1.00} % Change line spacing here, Palatino benefits from a slight increase by default

\makeatletter
\renewcommand{\@listI}{\itemsep=0pt} % Reduce the space between items in the itemize and enumerate environments and the bibliography

\renewcommand\abstractname{Résumé}
\renewcommand\refname{Références}
\renewcommand\contentsname{Table des matières}
\renewcommand{\maketitle}{ % Customize the title - do not edit title and author name here, see the TITLE block below
\begin{center} % Right align

\vspace*{25pt} % Some vertical space between the title and author name
{\LARGE\@title} % Increase the font size of the title

\vspace{125pt} % Some vertical space between the title and author name

{\large\@author} % Author name

\vspace{125pt} % Some vertical space between the author block and abstract
Dans le cadre du cours
\\INF8225 - Intelligence artificielle : Techniques probabilistes et d'apprentissage
\vspace{125pt} % Some vertical space between the author block and abstract
\\\@date % Date
\vspace{125pt} % Some vertical space between the author block and abstract

\end{center}
}

%----------------------------------------------------------------------------------------
%	TITLE
%----------------------------------------------------------------------------------------

\title{TP1 : Laboratoire 1} 

\author{\textsc{Raphael Lapierre 1644671} % Author
\vspace{10pt}
\\{\textit{École polytechnique de Montréal}}} % Institution

\date{02 Février 2016} % Date

%----------------------------------------------------------------------------------------

\begin{document}

\thispagestyle{empty}
\clearpage\maketitle % Print the title section
\pagebreak[4]
%----------------------------------------------------------------------------------------
%	En tête et pieds de page 
%----------------------------------------------------------------------------------------

\setlength{\headheight}{15.0pt}
\pagestyle{fancy}
\fancyhead[L]{INF8225}
\fancyhead[C]{}
\fancyhead[R]{TP1 - Laboratoire 1}
\fancyfoot[C]{\textbf{page \thepage}}

%----------------------------------------------------------------------------------------
%	ESSAY BODY
%----------------------------------------------------------------------------------------
\section*{Question 1}
Le code démontrant les exemples pratiques des trois phénomènes différents sont 
fournis avec l'archive. Le fichier à éxécuter est \textit{Question1.m}.
Des commentaires sont écrit à la console lors de l'éxécution pour plus d'informations.

\subsection*{Explaining Away}
L'explaining away est un phénomène qui se produit lorsque deux noeuds partagent des enfants.
Lorsqu'un des enfants qui partagé est observé, la modification d'observation sur un des parents
aura une influence sur l'autre. En effet, dans l'exemple de la question 1 montré en code, 
B n'a aucune influence sur F jusqu'à ce que leur enfant G soit observé. Dans ce cas, si on observe
B aussi, F voit sa probabilité modifiée. Les raisons expliquant B sont prise par G. On dit que 
G explain away F.

\subsection*{Serial Blocking}
Le serial blocking se produit dans une chaine de noeuds. Lorsqu'un des noeuds de la chaine
est observé, l'influence arrête de se transmettre au dela de ce noeuds des deux côtés de la
chaine. Toujours dans l'exemple de la question 1, on peut comprendre que la probabilité de 
D ne dépend plus de B si G et fixé.

\subsection*{Divergent Blocking}
Le divergent blocking se produit lorsque deux enfants possèdent un parent. Les enfants
on de l'influence entre eux parce que si on augmente les probabilité d'un enfant,
les probabilités de son parent augmente ce qui cause aussi une augmentation sur l'autre enfant
du même coup. Si on fixe le parent, les probabilités de l'enfant ne peuvent plus modifier celle
du parent et donc celle de l'autre enfant. Ils sont donc bloqués. Dans notre exemple, 
on fixe G pour montrer que D n'a plus d'influence sur F.

\section*{Question 2}
\subsection*{2.b)}
L'histogramme est montré lorsque le fichier matlab est executé. Il est difficile de formatter
le nom des axes pour montrer les situations qui mènent au résultat. Le résultat le plus à gauche
de l'histogramme est $[0,0,0,0,0]$ et celui à droite est $[1,1,1,1,1]$.

\subsection*{2.e)}
Voici l'équation pour calculer $P(J)$.
\begin{equation}
    P(J) = \sum\limits_{CTAM} P(C,T,A,J = V, M)
\end{equation}
\begin{equation}
    P(J) = \sum\limits_{CTAM} \prod\limits_{i = 1}^{n}P(V_{i}|parents(V_{i}))
\end{equation}
\begin{equation}
    P(J) = \sum\limits_{CTAM} P(C)P(T)P(A|C,T)P(M|A)P(J|A,T)
\end{equation}
On trouve donc 16 termes pour toutes les combinaisons de $CTAM$ qui une fois additionnés,
donne $P(J)$. Pour ce qui est de $P(C | J = V)$ on peut utiliser les équations suivantes.

\begin{equation}
    P(C|J = V) = \frac{P(C,J)}{P(J)}
\end{equation}
\begin{equation}
    P(C|J = V) = \frac{\sum\limits_{TAM} P(C = V,T,A,J = V, M)}{P(J)}
\end{equation}
\begin{equation}
    P(C|J = V) = \frac{\sum\limits_{TAM} \prod\limits_{i = 1}^{n}P(V_{i}|parents(V_{i}))}{P(J)}
\end{equation}
\begin{equation}
    P(C|J = V) = \frac{\sum\limits_{TAM} P(C)P(T)P(A|C,T)P(M|A)P(J|A,T)}{P(J)}
\end{equation}

\section*{Question 3}
La première étape pour effectuer l'algorithme EM dans l'exemple donné à la question 3 
est d'établir des probabilités au hazard. Une fois cette étape faite on peut commencer 
à appliquer les équations suivantes. On trouve les probabilités conjointes :

\begin{equation}
P(A|M,J) = \frac{\sum\limits_{MT} P(A,M,T)}{P(M,T)}
\end{equation}
\begin{equation}
P(M,J) = \sum\limits_{A} P(A)P(M|A)P(J|A)
\end{equation}

À l'aide des termes $P(M,J)$ ainsi que du nombre d'échantillion $N$ et $n_{i}$ pour le nième
exemple, on peut estimer le Log vraisemblance avec l'équation suivante :

\begin{equation}
Log Vraisemblance = \sum\limits_{i = 1}^{2^{2}} n_{i}\log(P)
\end{equation}

Dans cette équation il n'y a que quatres termes car les exemple ne dépendent que de M et J 
donc $2 * 2 = 4$.
Maintenant, à l'aide des probabilités $P(A|M,J)$, on peut trouver un nouveau $n_{i}$
en effectuant la multiplication : $n_{i} = n_{i} * P(M|A,J)$. Finalement, avec les nouveaux
$n_{i}$ on peut trouver les nouveaux paramètres pour le réseaux bayésien. Les itérations
continues jusqu'à convergence du Log vraimsemblance ou bien jusqu'à un nombre d'itération
maximal.

%----------------------------------------------------------------------------------------
\end{document}
